\title{Trending topics using the Twitter-API}
\author{
        Marco Pock-Steen Fraile \\
        Christian Palmh\o j Nielsen \\
        IT University of Copenhagen\\
}
\date{\today}

\documentclass[12pt]{article}

\begin{document}
\maketitle

\begin{abstract}
We have designed an algorithm for identifying trending topics in twitter..
\end{abstract}

\section{Introduction}
We wish to make use of the Twitter streaming API to implement our own version of trending topics using the Misra-Gries algorithm, and for the trending topics we would track how many unique tweets have been made per topic, using one of the algorithms for counting distinct elements. We could also count number of tweets for trending topics to compare actual tweets with unique tweets, but this is not algorithmically a challenge.
\newline\newline
Challenges would include finding a suitable k for the Misra-Gries that is a good trade-off between memory space and making sure we take into account the tweets inbetween the actual trending topics that might "flood" them out of the array.
\newline\newline
For the counting algorithm we similarly need to find a good hashing algorithm for tweets, and based on this find a good number for kth-minimum distances that gives us a realistic count for unique tweets.

\paragraph{Outline}
The remainder of this article is organized as follows.
Section~\ref{related work} gives account of previous work.
Our new and exciting results are described in Section~\ref{results}.
Finally, Section~\ref{conclusions} gives the conclusions.

\section{Twitter topics}

\section{Streaming algorithms}\label{related work}

\section{Algorithm design}
\subsection{Design}
\subsection{Running time}
\subsection{Satisfaction}

\section{Results}\label{results}
In this section we describe the results.

\section{Conclusions}\label{conclusions}
We worked hard, and achieved very little.

\bibliographystyle{abbrv}
\bibliography{simple}

\end{document}
This is never printed