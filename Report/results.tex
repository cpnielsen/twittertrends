\section{Results}\label{results}
In this section we describe the results of running the algorithm for a 1 hour period using the found values of \textit{k} for \textit{Misra-Gries} and \textit{KMV} respectively.
\\\\
$\begin{array}{ll}
    902: & \#peopleschoice (distinct: 922.0) \\
    575: & \#breakoutartist (distinct: 637.0) \\
    326: & \#cumannanyayap (distinct: 343.0) \\
    180: & \#teamfollowback (distinct: 236.0) \\
    160: & \#onedirection (distinct: 231.0) \\
    125: & \#camilafollowspree (distinct: 171.0) \\
    72: & \#gameinsight (distinct: 227.0) \\
    35: & \#tfbjp (distinct: 106.0) \\
    24: & \#android (distinct: 139.0) \\
    20: & \#androidgames (distinct: 137.0) \\
\end{array}$

\subsection{Comparison with Twitters trending topics}\label{twitter-result}
The results may differ from Twitters official trending topics for various reasons, which include but are not limited to the following: 
\\
\begin{enumerate}
	\item Twitters trend emphasizes novelty over popularity. This means that topics that have a high increase in frenquency over a short period of time are rated higher than topics that are always popular like \#android and \#gameinsight. 
	\item We only have access to a sample of all public tweets which is not guaranteed to be statistical accurate representation.
    \item For simplicity of implementation we only consider tweets that contains one or more tags. This does not seem to be the case for twitters own trending topics.
\end{enumerate}

By a quick comparison with the top-10 trending topics on Twitter with our results, it shows that there indeed exists some overlap of topics, but it is hard to define how much precisely.


