\subsection{Finding frequent items}\label{algo-frequent}
Ideally, we want to find the frequency of topics in $tweets$ in our input stream $\sigma = \langle a_{1}, a_{2},...,a_{m}\rangle$, and define a frequency vector $f = (f_{1},...,f_{n})$, where $f_{1} + ... + f_{n} = m$. 

To find the exact frequency of each topic would require that we store the exact number of times it has been seen, and with the vast number of topics seen (see section \ref{algo-data}) it would require a large amount of space to store. Instead, we can find the most frequent items seen over a period of time using the \textit{Misra-Gries} algorithm.\cite{Amit}
\newline

From the algorithm we have $f_j - \frac{m}{k} \leq \hat{f_j} \leq f_j$. From this we can derive that any item with a $\hat{f_j} > \frac{m}{k}$ will be in our map $A[j]$ with a positive score, and we simply need to make sure the topics we wish to capture for a given time period are above the $\frac{m}{k}$ threshold.

\subsubsection{Determine k and m for Misra-Gries}
In order for us to determine a suitable $k$ we would first have to find $m$ for a given timeframe. We wrote a piece of code that for 1 hour and 24 hours would count accurately \textit{a)} The total number of tweets, \textit{b)} The number of distinct topics and \textit{c)} The number of occurrances for each topic. We ran this a few times on different days to get an average on the number of tweet and occurances of the top 10 topics. An example of each sampling (1 hour and 24 hour respectively) can be seen below.\newline

Total number of tweets in 60 minutes: 26733\\

$\begin{array}{ll}
    454: & \#PeoplesChoice \\
    446: & \#StarAc \\
    331: & \#IHaveACrushOn \\
    316: & \#tbt \\
    265: & \#gameinsight \\
    251: & \#RT \\
    240: & \#TeamFollowBack \\
    223: & \#LouisAndBoris \\
    146: & \#musicfans \\
    140: & \#FF \\
\end{array}$
\\
\\

Total number of tweets in 1440 minutes: 534408\\

$\begin{array}{ll}
    17064: & \#FF \\
    7271: & \#gameinsight \\
    6589: & \#PeoplesChoice \\
    4775: & \#TeamFollowBack \\
    4433: & \#RT \\
    3692: & \#androidgames \\
    3522: & \#TFBJP \\
    3510: & \#MessageToMyEx \\
    3456: & \#ff \\
    3370: & \#android \\
\end{array}$
\\
\\
Given this data, we can determine an approximation for $k$. We want to include at least the top 10 topics for the time periods, so we maintain two seperate data structures for each. For the 60 minute period, we want topics with a minimum frequency of \textbf{140} occurrences over \textbf{26733} total events. This means we have $k = \frac{26733}{140} \approx 190$. If we take the average of all our samplings, we find the average to be $k = 155$ for a 24-hour period, and $k = 165$ for a 1-hour period.
\\
\\
We discuss the results and our conclusions in section \ref{trending-discussion}
