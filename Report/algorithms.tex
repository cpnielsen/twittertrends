\section{Streaming algorithms}\label{related work}
\subsection{Introduction}\label{algo-intro}
\begin{quote}\textit{
Twitter and other social networks are structured to accomodate personal communication across large networks of friends, and a such produce enormous amounts of data. The open availability of this data through developer APIs makes it's an interesting source for useful real-time information extraction using streaming algorithms \cite{genderprediction}. 
}
\end{quote}

This section introduces the concepts of algorithms that can compute some function of a massively long input stream $\sigma$ such as all public available tweets. In our model this is formalized as a sequence $\sigma = \langle a_{1}, a_{2},...,a_{m}\rangle $, where the elements of the sequence (in our case $tweets$) are drawn from the universe $[n] =: \{1, 2,..., n\}$. Note the two size parameters: the stream length, $m$, and the universe size, $n$.

\subsection{Goal}\label{algo-goals}
Our goal is to find the trending topics on twitter over a given time period. The idea is that our algorithm will be optimized for different intervals, eg. 1-hour and 24-hour periods, and that at any given time it can be queried to get the actual trending topics for previous interval (assuming the stream has been running for that minimum amount of time).
Our goal will be to process the input stream using a small amount of space $s$, i.e., to use s bits of random-access working memory. Since $m$ and $n$ are to be thought of ``huge'' we want to make $s$ much smaller than these. Ideally we want to achieve $s$ = O(log $m$ + log $n$), because this is the amount of space needed to store a constant number of elements from the stream and a constant number of counters that can count up to the length of the stream. \cite{Amit}

Specifically we want to analyze the Twitter input stream to find tags that that have a high occurence over some period of time, and be able to return these e.g. as a top-10 list for both 1 hour and 24 hour periods. Besides returning the most popular topics, we wish to be able to say something about if the topic has a high frequency because many different people have mentioned it, or it has been retweeted a lot of times.

\subsection{Finding frequent items}\label{algo-frequent}
Ideally, we want to find the frequency of topics in $tweets$ in our input stream $\sigma = \langle a_{1}, a_{2},...,a_{m}\rangle$. We define a frequency vector $f = (f_{1},...,f_{n})$, where $f_{1} + ... + f_{n} = m$. 


To find the exact frequency of each topic would require that we store the exact number of times it has been seen, and with the vast number of topics seen (see section \ref{frequent-determine}) it would require a large amount of space to store. Instead, we can find the most frequent items seen over a period of time using the \textit{Misra-Gries} algorithm.\cite{Amit}
\newline

From the algorithm description we have $f_j - \frac{m}{k} \leq \hat{f_j} \leq f_j$. From this we can derive that any item with a $\hat{f_j} > \frac{m}{k}$ will be in our map $A[j]$ with a positive score, and we simply need to make sure the topics we wish to capture for a given time period are above the $\frac{m}{k}$ threshold. We will discover $m$ and derive $k$ from it in the following section.

\subsubsection{Determine k and m for Misra-Gries}
\label{frequent-determine}
In order for us to determine a suitable $k$ we will first have to find $m$ for a given timeframe. We wrote a piece of code that for 1 hour and 24 hours would count accurately \textit{a)} The total number of tweets, \textit{b)} The number of distinct topics and \textit{c)} The number of occurrances for each topic. We ran this a few times on different days to get an average on the number of tweet and occurances of the top 10 topics. An example of each sampling (1 hour and 24 hour respectively) can be seen below.\newline

Total number of tweets in 60 minutes: 26733, with 19117 unique topics\\

$\begin{array}{ll}
    454: & \#PeoplesChoice \\
    446: & \#StarAc \\
    331: & \#IHaveACrushOn \\
    316: & \#tbt \\
    265: & \#gameinsight \\
    251: & \#RT \\
    240: & \#TeamFollowBack \\
    223: & \#LouisAndBoris \\
    146: & \#musicfans \\
    140: & \#FF \\
\end{array}$
\\
\\

Total number of tweets in 1440 minutes: 534408, with 214345 unique topics\\

$\begin{array}{ll}
    17064: & \#FF \\
    7271: & \#gameinsight \\
    6589: & \#PeoplesChoice \\
    4775: & \#TeamFollowBack \\
    4433: & \#RT \\
    3692: & \#androidgames \\
    3522: & \#TFBJP \\
    3510: & \#MessageToMyEx \\
    3456: & \#ff \\
    3370: & \#android \\
\end{array}$
\\
\\
Given this data, we can determine an approximation for $k$. We want to include at least the top 10 topics for the time periods, so we maintain two seperate data structures for each. For the 60 minute period, we want topics with a minimum frequency of \textbf{140} occurrences over \textbf{26733} total events. This means we have $k = \frac{26733}{140} \approx 190$. If we take the average of all our samplings, we find the average to be $k = 155$ for a 24-hour period, and $k = 165$ for a 1-hour period. We assumed that $k$ would be larger for the 24 hour period, but as some topics keep repeating frequently spread over a day, they actually require a smaller k for bigger intervals. Also, our samples are not for longer periods (months), so this might not be completely accurate.
\\
\\
We discuss the results and our conclusions in section \ref{trending-discussion}


\subsection{Finding distinct values}\label{algo-distinct}
K-minimum values (KMV) is is a probalistic distinct value counting algorithm, that is intuitive and easy to implement \cite{kmv}. Suppose we have a good hash function that return evenly distributed values in the hash space $[0-1]$, then you could estimate the number of distinct values you have seen by knowing the average spacing between values in the hash space. The main challenge is to find a good hash function, and to select the number of minimum $k$ values on which to approximate the average spacing. If the hash values were indeed evenly distributed, we could keep just keep track of the minimum value, a get a good estimate of distinct values. However taking only one value opens up to a lot of variance and would rely heavily upon the``goodness'' of the hash function. In order to improve this Bar-Yossef\cite{Bar-Yossef} suggests keeping the k-smallest values, to give a more realistic estimate.
   
Other examples of the usages of data stream algorithms are described in this chapter.
\subsection{Implementation of algorithms}

\subsubsection{Misra-Gries Algorithm}\label{misra-gries}
The algorithm first initializes a dictionary with $k$ number of values. The keys in the dictionary are elements seen in the stream, and the value are counters associated with the elements. Then there is a process function that is executed each time we see a new element. If a new element is already in the dictionary, it's value will be increased by 1, otherwise if the number of elements in $A$ is less than $k$, the element will be inserted an its value set to 1. If the length of $A$ is equal to $k$, all values are decreased by 1, and removed if the value is equal to 0 \cite{Amit}. Finally we return the key value pairs with the highest frequencies.

We use Misra-Gries to find frequent $topics$ in our data stream of tweets using a one-pass algorithm, see pseudo-code \ref{misra-pseudo} for details.
\begin{algorithm}
\caption{Misra-Gries Algorithm}\label{misra-pseudo}
\begin{algorithmic}[1]
\State $A\gets$ Initialize map
\State $KMV\gets$ Initialize $KMV_{j}$ set for topic $j$
\State $h\gets$ Hash function maps to $[0..1]$
\Statex
\Function{OnTweet}{$j$, $m$}
\If {$j \in keys(A)$}\Comment{Topic already seen}
    \State $A[j] \leftarrow A[j] + 1$
    \State $KMV_{j}$ add $h(m)$\Comment{Add hash value to set}
\ElsIf {$|keys(A)| < k-1 $}
    \State $A[j] \leftarrow 1$
    \State $KMV_{j}$ add $h(m)$
\Else
    \For{$l \in keys(A)$}
        \State {$A[l] \leftarrow A[l] - 1$}
        \If {$A[l] = 0$} 
        \State remove $A[l]$
        \State remove $KMV_{l}$
        \EndIf
    \EndFor
\EndIf
\EndFunction
\Statex
\Function{KMV-Distinct}{$k$, $j$}
\State $S\gets$ sorted KMV set
\State $distinct = k-1/S[k]$
\EndFunction
\end{algorithmic}
\end{algorithm}
\subsubsection{Data structure}
When a new tweet is seen the algorithm has to determine if the contained topic has been seen before. We use a dictionary with key-value pair as topic-frequency. The dictionary provides $O(1)$ time lookup and increment operations. When we decrement all elements by one we iterate over the dictionary which takes $O(k)$ time. Finally we sort the dictionary, but since it's limited to $k$ items, this is not a very time consuming operation.
\subsubsection{Data set}\label{algo-data}

\subsubsection{Analysis of algorithm}\label{algo-analysis}
Misra-Gries uses one pass with worst case running time $O(k)$ for each element of the stream $[n]$. Insertion and increments of a topic takes $O(1)$ time if the length of the dictionary is less than $k$. However the worst case scenario would decrement all values of the dictionary every time the dictionary is full, a maximum of $\frac{k}{n}$ times, so we can argue that the amortized running time for each process is $O(1)$.
